\section{Introduction}
\label{sec:intro}

The search of the Higgs boson is one of the primary goals of the ATLAS detector. Collisions at center of mass energies of $\sqrt{s}$=7 and 8TeV yielded evidence for several decay modes of the Higgs and a combination between the ATLAS and CMS experimental results claims evidence for a new particle compatible with the Standard Model Higgs Boson decaying into taus. The first collisions at $\sqrt{s}=13$TeV lead to the possibility of strengthening this evidence with a `pure' result derived from each detector independently. The analysis laid out by the ATLAS group features a 'cut based' analysis whereby the final state topology of the Higgs is exploited to enhance any new signals against their backgrounds. 

The primary irreducible background to Higgs Searches with Taus is the decay of the Z$^0$ boson into a tau anti-tau pair. The analysis targets kinematic variables such as the resonant mass of the tau pair to try to remove as many Z type events whilst retaining as many Higgs type events as possible. This means that the remaining Z$\rightarrow\tau\tau$ events are in a highly contrived region of phase-space where the theoretical prediction of the events is highly important to the sensitivity of the analysis. 

In view of the complexity of the
relevant event properties, in the past the ATLAS collaboration endeavoured to rely as little as possible on simulation. However Z$\rightarrow\tau\tau$ model cannot be obtained directly from the collision data due to background contributions, e.g.  from events with other objects misidentified as tau decays. Events with two muons can be `embedded' with simulated tau decays such that kinematic quantities can be preserved. However such a process requires a large Z$\rightarrow\mu\mu$ data set and extensive validation.

In practice, particle physics analyses use Monte-Carlo (MC) generators to compare predictions from theory to data. An extensive system of simulation and reconstruction mimics the effects of the detector such that theoretical models can be compared directly with physics objects in data. A full description of all relevant processes in simulated MC is considered by many to be the only way that a process in the ATLAS detector can be observed. Moreover, it allows us to produce new events for testing and refining our analysis regardless of the performance of the LHC and ATLAS detector. As such the production of high statistic, high prescision MC modeling of the Z$\rightarrow\tau\tau$ process is key to the search for Higgs Bosons in the first $\sqrt{s}=13$TeV data with the ATLAS detector.

The Z$\rightarrow\tau\tau$ process in MC is highly complex. Due to the properties of the Z$^0$ and the tau almost all observable quantities are correlated. To reduce Z and DY contribution to the higgs analysis the signal regions either have a high transverse mass (possibly with additional jets) or explicity have at least 2 additional jets. The production of these additional jets requires a very large number of additional QCD and EW production modes must be calculated.  

This note describes the generators used for producing simulated Z+jets events as used in the ATLAS $H\rightarrow\tau\tau$ analysis and demonstrates its performance using the first X data at $\sqrt{s}=13$TeV as collected by the ATLAS detector. 3 possible channels are considered here. In the case that both taus decay leptonically the Z$\rightarrow\tau\tau$ component is enhanced when one tau decays into an electron and one into a muon. In the case where both leptons measured have the same flavour the Z$\rightarrow\tau\tau$ component is surpressed and the Z$\rightarrow\ell\ell$ component can be used to cleanly asses the MC performance of the Jet activity associated with these events. The case where one tau decays hadronically and one leptonically is predicted to give the greatest quantity of Z$\rightarrow\tau\tau$ events and is therefore also considered.